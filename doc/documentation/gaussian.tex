
\documentclass{article}
\input{myinclude}
\usepackage[numbers]{natbib}

\newcommand{\half}{\tfrac{1}{2}}

\begin{document}

\title{gaussian<T>}
\maketitle

\section*{Description}

\texttt{Gaussian<T>} is a kernel function. Specifically, it is a Mercer Kernel. If k is an object of class \texttt{gaussian<T>}, and u and v are objects of class T, then k(u,v) returns
%
$$k(u,v) = \textrm{exp}( \half \sigma^{-2} \Vert u-v \Vert^2 ) $$
%
where $\sigma$ is the width of the kernel. 


\section*{Example}


\highlightcpp{}
\begin{verbatim}
vector< double > u(10);
vector< double > v(10);
gaussian< vector< double > > kernel(1.0);
cout << kernel( u, v ) << endl;
\end{verbatim}


\section*{Definition}

Defined in the KML header \href{gaussian.hpp}{gaussian.hpp}.


\section*{Template Parameters}

\begin{tabular}{lll}
\textbf{Parameter} & \textbf{Description} & \textbf{Default} \\ 
T & The gaussian argument type \\ 
\end{tabular}


\section*{Model of}

Mercer Kernel, Default Constructable, Copy Constructable


\section*{Type requirements}

T must be a vector type or a numeric type; distance_squared<T> should evaluate.


\section*{Members}

\begin{tabular}{lll}
\textbf{Member} & \textbf{Where defined} & \textbf{Description} \\ 
\texttt{gaussian()} & \href{http://www.sgi.com/tech/stl/DefaultConstructible.html}{Default Constructable} & The default constructor \\
\texttt{result_type} & Input value & The type of the result: \texttt{input_value<T>} \\
\end{tabular}

\section*{Notes}

\section*{See also}

The Mercel Kernel overview, Adaptable Binary Function, 
\href{research/kml/documentation/linear.html}{linear}, 
\href{research/kml/documentation/polynomial.html}{polynomial}, 
\href{research/kml/documentation/sigmoid.html}{sigmoid}

\bibliographystyle{unsrtnat}
\bibliography{/home/rutger/documents/bibliography/references}
\end{document}



