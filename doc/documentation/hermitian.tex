\documentclass{article}
\input{myinclude}
\usepackage[numbers]{natbib}

\newcommand{\half}{\tfrac{1}{2}}

\begin{document}

\title{hermitian<T,N>}
\maketitle

\section*{Description}

\texttt{Hermitian<T,N>} is a family of Kernel Functions. Specifically, it is a family of 
\href{\kmlroot/mercer_kernel.html}{Mercer Kernels}. 
The N-th Hermitian kernel is equal to the N-th order derivative of the Gaussian kernel. 
They are named after Hermite polynomials, which may be used to identify the derivatives of the Gaussian kernels. 

If k is an object of class \texttt{hermite<T,N>}, u and v are objects of class T, and N is a integral constant, then k(u,v) returns
%
\begin{equation}
D^{n}k(u,v)=(-\sqrt{2}\sigma)^{-n}H_{n}((\sqrt{2}\sigma)^{-1}(u-v))\exp(-\half\sigma^{-2}\left\Vert u-v\right\Vert^{2}).
\end{equation}
%
where $H_n$ is Rodrigues' formula for Hermite polynomials
\begin{equation}
H_{n}(x)=(-1)^{n}\exp(x^{2})D^{n}\exp(-x^{2}).
\end{equation}
%
Figure~\ref{figure:hermitian_kernel} shows the Hermitian kernel for N=0 through N=5 for a scalar input type.

\begin{figure}
\fullRplot{hermitian_kernel}
\caption{Hermitian kernels, parametrised by $\sigma=1$, with N=0 at the top left through N=5 at the bottom right.}
\label{figure:hermitian_kernel}
\end{figure}


\section*{Example}


\highlightcpp{}
\begin{verbatim}
vector< double > x(10);
vector< double > v(10);
hermitian< vector< double >, 2 > kernel(1.0);
cout << kernel( x, v ) << endl;
\end{verbatim}


\section*{Definition}

Defined in the KML header \href{\kmlsvnroot/kml/hermitian.hpp}{kml/hermitian.hpp}.


\section*{Template Parameters}

\begin{tabular}{lll}
\textbf{Parameter} & \textbf{Description} & \textbf{Default} \\ 
\hline
T & The hermitian argument type \\ 
N & The order of the hermitian \\ 
\end{tabular}


\section*{Model of}

\href{\kmlroot/mercer_kernel.html}{Mercer Kernel}

\section*{Type requirements}

T must be a vector type or a scalar type.


\section*{Members}

\begin{tabular}{lll}
\textbf{Member} & \textbf{Where defined} & \textbf{Description} \\ 
\hline
\texttt{hermitian()} & \href{http://www.sgi.com/tech/stl/DefaultConstructible.html}{Default Constructible} & The default constructor \\
\texttt{result_type} & Input value & The type of the result: \texttt{input_value<T>} \\
\end{tabular}

\section*{Notes}

\section*{See also}

\href{\kmlroot/mercer_kernel.html}{Mercer Kernel},
\href{\kmlroot/linear.html}{linear},
\href{\kmlroot/linear.html}{gaussian},
\href{\kmlroot/linear.html}{polynomial},
\href{\kmlroot/linear.html}{sigmoid}

\bibliographystyle{unsrtnat}
\bibliography{/home/rutger/documents/bibliography/references}
\end{document}



