
\documentclass{article}
\input{myinclude}

\begin{document}

\title{Obtaining KML}
\maketitle

\section*{Dependencies}

The Kernel-Machine Library builds on top of the libraries below.

\begin{itemize}
\item \textbf{Boost.} Functionality of several \href{http://www.boost.org}{Boost libraries} is 
in use throughout the Kernel-Machine Library. Boost provides
an excellent set of C++ libraries which reduces the amount of code and boosts the quality of the overall implementation.
The Kernel-Machine Library is built and tested against version 1.32.0.

\item \textbf{Boost Numeric Bindings.} A part of Boost's sandbox CVS repository is required, namely the numeric bindings. 
This is a software package that combines the flexibility of the C++ data types (such as that of 
\texttt{std::vector<>} or \texttt{ublas::matrix<>}) with the computational 
efficiency of a BLAS and LAPACK.  

\item \textbf{ATLAS.} This is a very efficient basic linear algebra system (BLAS) \citep{whaley01automated}, located at
\href{http://math-atlas.sourceforge.net}{Sourceforge}. ATLAS can empirically finetune its code to make optimal use of your CPU 
including its extensions such as SSE, 3DNow! and AltiVec. It also contains several LAPACK routines. 

\end{itemize}

\section*{Download}
\label{section:download}

The latest release of the Kernel-Machine Library is 0.1. However, this is not the recommended version; you are encouraged to
download the latest developer version through  
\href{http://www.terborg.net/cgi-bin/viewcvs.cgi/trunk/kml/}{viewCVS} until 0.2 is released. 
Version 0.1 can still be downloaded, either as a tarball 
from \href{research/kml/kml-0.1.tar.gz}{kml-0.1.tar.gz} or as a zipped archive
from \href{research/kml/kml-0.1.zip}{kml-0.1.zip}.

I have created a tarball of 
\href{research/kml/boost\_bindings.tar.gz}{boost\_bindings.tar.gz}
 and a zipfile 
\href{research/kml/boost\_bindings.zip}{boost\_bindings.zip}%
, which are on basis of Boost's sandbox CVS of April 3rd, 2005.


The development version of the Kernel-Machine Library can be checked out anonymously from 
the trunk directory of the subversion repository located at 
\href{http://www.terborg.net/svn/trunk/kml/}{http://www.terborg.net/svn/trunk/kml/}. By using the following command
\begin{verbatim}
svn checkout http://www.terborg.net/svn/trunk/kml
\end{verbatim}
a fresh up-to-date KML directory will be obtained. 


\bibliographystyle{unsrtnat}
\bibliography{/home/rutger/documents/bibliography/references}
\end{document}








