
\documentclass{article}
\input{myinclude}
\usepackage[numbers]{natbib}


% algorithm display
\usepackage{algorithm}
\usepackage{algorithmic}

% for the figures of xfig
\usepackage{color}

% linebreaks for urls
\usepackage{url}

% some math definitions
\newcommand{\x}{x}
\newcommand{\sv}{Q}
\newcommand{\svm}{Q}
\newcommand{\sve}{q}
\newcommand{\nsve}{\mathbf{e}}
\newcommand{\nonsv}{A}
\newcommand{\nonsvm}{A}
\newcommand{\nonsve}{a}
\newcommand{\origin}{0}
\newcommand{\gramm}{G}
\newcommand{\ce}{R}
\newcommand{\coordm}{R}



\begin{document}


\title{Incomplete Cholesky Factorisation}
\maketitle

\tableofcontents

\section*{Description}

The incomplete Cholesky is used in conjunction with 
Support Vector Machines in \citep{fine01efficient}, ...


In this section we will give a geometric interpretation to 
the incomplete cholesky factorisation.

We will start with subsection~\ref{subsection:two_circle_intersection},
where we show that it is possible to project a vector on another vector
by means of a two-circle intersection. We incrementally extend the
idea to a three-sphere intersection in subsection~\ref{subsection:three_sphere_intersection},
and conclude this section with a generic projection operator
obtained by an associated $n$-hypersphere intersection.

$$A=QR$$

where matrix $\nonsv$ is has columns with mapped defined by be a
shorthand notation for our set of data $\mathcal{D}$ mapped into
$\mathcal{H}$,%
%
\begin{equation}
\nonsv=\{\nonsve_{1},...,\nonsve_{n}\}=\{\Phi(\x_{1}),...,\Phi(\x_{n})\}.
\label{equation:set_to_origin}
\end{equation}
%
Obviously, the inner product of vectors $\nonsve_{i}$ in $\nonsv$
may be expressed in terms of kernels,
%
\begin{equation}
\nonsve_{i}^{\mathrm{T}}\nonsve_{j}=\Phi(\x_{i})^{\mathrm{T}}\Phi(\x_{j})=k(\x_{i},\x_{j}),
\label{equation:inner_product_V}
\end{equation}%
%
whereas the the squared length of a vector $\nonsve_{i}$ may be expressed
as
%
$$\Vert\nonsve_{i}\Vert^{2}=\Vert\Phi(\x_{i})-\origin\Vert^{2}=k(\x_{i},\x_{i})$$
%
where $\origin$ denotes the origin of feature space $\mathcal{H}$.


\subsection{The Two-Circle Intersection\label{subsection:two_circle_intersection}}

Normally, one begins the Gram-Schmidt procedure by projecting one
vector onto another. 

The key to the geometric interpretation of the incomplete
cholesky factorisation is to re-express the projection
of vector $\nonsve_{2}$ onto basis vector $\sve_{1}:\nonsve_{1}/\ce_{1,1}$
as presented in Equation~\ref{equation:standard_projection} in terms
of distances. Figure~\ref{figure:two_circle_intersection} shows two circles
on a two-dimensional plane defined by the span of $\sve_{1}$ and
$\nonsve_{2}$. %
\begin{figure}
\begin{center}
\figwithmath{two_circle_intersection}
\end{center}
\caption{The projection of vector $\nonsve_{2}$ on vector $\sve_{1}$, and
its associated two-circle intersection problem.%
\label{figure:two_circle_intersection}}
\end{figure}
%
The centres of the circles are placed at the origin and the head
of $\nonsve_{1}$. They have radii such that they go through the head
of vector $\nonsve_{2}$. This means that the left circle has a radius
of $\Vert\nonsve_{2}\Vert$ equal to the length of vector $\nonsve_{2}$,
and that the right circle has a radius of $d(\nonsve_{1},\nonsve_{2})$.
The \emph{lens} of any circle-circle intersection is perpendicular
to the line through its centres. In this case this means that the
intersection is perpendicular to $\sve_{1}$, thereby fulfilling the
necessary property of orthogonality of a projection. 

The simple equation of a circle is known as $x^{2}+y^{2}=r^{2}$,
.... 

The two-circle intersection problem as shown in Figure~\ref{figure:two_circle_intersection}
may be expressed with the following system of equations. 

\begin{eqnarray*}
\ce_{1,2}^{2}+d_{\perp}^{2}(\sv,\nonsve_{2}) & = & \Vert\nonsve_{2}\Vert^{2}\\
(\ce_{1,2}-\ce_{1,1})^{2}+d_{\perp}^{2}(\sv,\nonsve_{2}) & = & d^{2}(\nonsve_{1},\nonsve_{2})\end{eqnarray*}
The system of equations consists of two equations with two unknowns,
coordinate $\ce_{1,2}$ and perpendicular distance of vector $\nonsve_{2}$
to the basis defined by $\sv$, $d_{\perp}(\sv,\nonsve_{2})$, and
they can be solved rather straightforwardly. The solution is given
below. \begin{eqnarray*}
\ce_{1,2} & = & \frac{\ce_{1,1}^{2}+\Vert\nonsve_{2}\Vert^{2}-d^{2}(\nonsve_{1},\nonsve_{2})}{2\ce_{1,1}}=\frac{\nonsve_{1}^{T}\nonsve_{2}}{\ce_{1,1}}\\
d_{\perp}(\sv,\nonsve_{2}) & = & \sqrt{\Vert\nonsve_{2}\Vert^{2}-\ce_{1,2}^{2}}\end{eqnarray*}
the upper equation .. may be simplified to .. 

Actually, there is a second solution to $d_{\perp}(\sv,\nonsve_{2})$,
one having a negative square root. Let us ignore that solution, thereby
obeying the axiom of positiveness of distances. The resulting projection
of $\nonsve_{2}$ on $\sve_{1}$ in terms of $\sve_{1}$ may be obtained
using the solution above, \[
\hat{\nonsve}_{2}=\ce_{1,2}\sve_{1}.\]
Remarkably, we have obtained more information than the projection
of $\nonsve_{i}$ only: the perpendicular distance $d_{\perp}(\sv,\nonsve_{2})$
of basis $\sv$ to vector $\nonsve_{2}$. We will come back to this
issue later on.

\begin{eqnarray*}
\ce_{2,2} & = & d_{\perp}(\sv,\nonsve_{2})\\
\sve_{2} & = & (\nonsve_{2}-\hat{\nonsve}_{2})/\ce_{2,2}\end{eqnarray*}



\subsection{The Three-Sphere Intersection\label{subsection:three_sphere_intersection}}

Before discussing the arbitrary-dimensional case, let us first consider
a projection of a vector $\nonsve_{i}$ onto a two-dimensional plane.
Assume we already have two top-notch orthonormal basis vectors 

Let the current basis vectors be defined by $\sv=(\sve_{1},\sve_{2})$,
.... 

\begin{eqnarray*}
\sve_{1} & : & \nonsve_{1}/\ce_{1,1}\\
\sve_{2} & : & (\nonsve_{2}-\ce_{1,2}\sve_{1})/\ce_{2,2}\end{eqnarray*}
Basis vector $\sve_{2}$ is orthogonal to basis vector $\sve_{1}$,
possibly determined by a two-circle intersection presented in the
preceding subsection. 

Figure~\ref{figure:three_sphere_intersection} illustrates the problem.
It may be solved by the solution to a three-sphere intersection problem.
%
\begin{figure}
\begin{center}\figwithmath{three_sphere_intersection}\end{center}


\caption{The projection of $\nonsve_{i}$ on plane $\sv$ may be tackled by
a three-sphere intersection problem.\label{figure:three_sphere_intersection}}
\end{figure}
 Let the spheres be situated in a three-dimensional space defined
by the span of $\sve_{1}$, $\sve_{2}$ and $\nonsve_{3}$, and let
the centre of the first sphere be located at $\origin\in\mathcal{H}$,
and that of the other two at the heads of vectors $\nonsve_{1}$ and
$\nonsve_{2}$. Let the spheres be scaled in such way that they go
through the head of vector $\nonsve_{3}$, having radii $\Vert\nonsve_{3}\Vert$,
$d(\nonsve_{1},\nonsve_{3})$ and $d(\nonsve_{2},\nonsve_{3})$. 

The system of equations of a two-circle intersection problem may be
extended to match a three-sphere intersection problem. \begin{eqnarray*}
\ce_{1,3}^{2}+\ce_{2,3}^{2}+d_{\perp}^{2}(\sv,\nonsve_{3}) & = & \Vert\nonsve_{3}\Vert^{2}\\
(\ce_{1,3}-\ce_{1,1})^{2}+\ce_{2,3}^{2}+d_{\perp}^{2}(\sv,\nonsve_{3}) & = & d^{2}(\nonsve_{1},\nonsve_{3})\\
(\ce_{1,3}-\ce_{1,2})^{2}+(\ce_{2,3}-\ce_{2,2})^{2}+d_{\perp}^{2}(\sv,\nonsve_{3}) & = & d^{2}(\nonsve_{2},\nonsve_{3})\end{eqnarray*}
 In this case we have three equations with three unknowns, being $\ce_{1,3}^{2}$,
$\ce_{2,3}^{2}$, and $d_{\perp}(\sv,\nonsve_{3})$. Although the
solution is less trivial than that for the two-circle intersection
problem, it remains rather straightforward. It is given below.\begin{eqnarray*}
\ce_{1,3} & = & \frac{\nonsve_{1}^{T}\nonsve_{3}}{\ce_{1,1}}\\
\ce_{2,3} & = & \frac{\nonsve_{2}^{T}\nonsve_{3}-\ce_{1,2}\ce_{1,3}}{\ce_{2,2}}\\
d_{\perp}(\sv,\nonsve_{3}) & = & \sqrt{\Vert\nonsve_{3}\Vert^{2}-\ce_{1,3}^{2}-\ce_{2,3}^{2}}\end{eqnarray*}
In comparison to the solution to a two-circle intersection problem,
we make the observation that the solution to $\ce_{\nonsve_{i},1}$
has not changed shape. We have deliberately expressed the solution
to $d_{\perp}(\sv,\nonsve_{i})$ in terms of $\ce_{1,3}$ and $\ce_{2,3}$
to improve clarity.

Finally, the projection of vector $\nonsve_{i}$ on the plane brought
forth by $\sv$ is expressed with \[
\hat{\nonsve}_{3}=\ce_{1,3}\sve_{1}+\ce_{2,3}\sve_{2}.\]
And one could create a basis vector by simply substituting \begin{eqnarray*}
\ce_{3,3} & = & d_{\perp}(\sv,\nonsve_{3})\\
\sve_{3} & = & (\nonsve_{3}-\hat{\nonsve}_{3})/\ce_{3,3}\end{eqnarray*}
.


\subsection{The $n$-Hypersphere Intersection\label{subsection:hypersphere_intersection}}

Let us consider generic case of the projection of a vector $\nonsve_{i}$
on a hyperplane spanned by $\sv$. We are given $(n-1)$ ordered orthogonal
basis vectors \begin{eqnarray*}
\sve_{1} & : & \nonsve_{1}/\ce_{1,1}\\
\sve_{2} & : & \left(\nonsve_{2}-\ce_{1,2}\sve_{1}\right)/\ce_{2,2}\\
\sve_{3} & : & \left(\nonsve_{3}-\ce_{1,3}\sve_{1}-\ce_{2,3}\sve_{2}\right)/\ce_{3,3}\\
\vdots &  & \vdots\\
\sve_{n-1} & : & \left(\nonsve_{n-1}-\sum_{j=1}^{j=n-2}\ce_{j,n-1}\sve_{j}\right)/\ce_{n-1,n-1}\end{eqnarray*}
that span the hyperplane in which patterns the origin $\origin\in\mathcal{H}$
and vectors $\sve_{1},\ldots,\sve_{n-1}$ are positioned. Generally,
this projection may be determined by an $n$-hypersphere intersection.
The generic system of equations is given by \begin{eqnarray*}
\ce_{1,n}^{2}+\ce_{2,n}^{2}+\ldots+d_{\perp}^{2}(\sv,\nonsve_{n}) & = & \Vert\nonsve_{i}\Vert^{2}\\
(\ce_{1,n}-\ce_{1,1})^{2}+\ce_{2,n}^{2}+\ldots+d_{\perp}^{2}(\sv,\nonsve_{n}) & = & d^{2}(\nonsve_{1},\nonsve_{n})\\
 & \vdots\\
(\ce_{1,n}-\ce_{1,n-1})^{2}+\ldots+(\ce_{n-1,n}-\ce_{n-1,n-1})^{2}+d_{\perp}^{2}(\sv,\nonsve_{n}) & = & d^{2}(\nonsve_{n-1},\nonsve_{n})\end{eqnarray*}
for any $n$. We can solve this system of $n$ equations with the
following solutions

\begin{eqnarray}
\ce_{j,n} & = & \frac{\nonsve_{j}^{T}\nonsve_{n}-\sum_{p=1}^{p=j-1}\ce_{p,j}\ce_{p,n}}{\ce_{j,j}}\label{equation:coord_update}\\
d_{\perp}(\sv,\nonsve_{n}) & = & \sqrt{\Vert\nonsve_{n}\Vert^{2}-\sum_{j=1}^{j=n-1}\ce_{j,n}^{2}}\label{equation:perp_distance}\end{eqnarray}
The way the solution is written down, requires that the computations
need to be executed consecutively in the order of the basis vectors,
i.e., $\ce_{\nonsve_{i},j-1}$ needs to be computed before $\ce_{\nonsve_{i},j}$
for each $j>1$. The projection of $\nonsve_{i}$ on the hyperplane
defined by basis $\sv$ may be obtained by evaluating\[
\hat{\nonsve}_{i}=\sum_{j=1}^{j=n-1}\ce_{j,i}\sve_{j}.\]


\subsection{The Algorithm\label{subsection:max_perp_algorithm}}

In this section, we will introduce an orthogonalisation process that
utilises the projection operator introduced in the preceding
section. This is the pseudo code for the incomplete cholesky factorisation.
%
\begin{algorithm}
\begin{algorithmic}

\FOR{$i=1:n$}

\STATE $c(i)=a_i^T a_i$

\ENDFOR

\STATE $k = \mathrm{max~element} \{c(1),\ldots,c(n)\}$

\STATE $\tau=c(k)$; $r=0$

\WHILE{$\tau>0$ \textbf{and} $r<n$}

\STATE $r=r+1$

\STATE $c(r) \leftrightarrow c(k)$; $R(1$:$r$-$1,r) \leftrightarrow R(1$:$r$-$1,k)$

\STATE $R(r,r) = \sqrt{\tau}$

\STATE $R(r,r$+$1$:$n) = R(1$:$r$-$1,r$+$1$:$n)^T R(1$:$r$-$1,r) $

\FOR{$i$ in $r$+$1$:$n$}

\STATE $R(r,i) = ( a_k^T a_i - R(r,i) ) / \sqrt{\tau} $

\STATE $c(i) = c(i) - R(r,i)^2$

\ENDFOR

\IF{compute Q}

\STATE $Q(1$:$m,r) = ( A(1$:$m,k) - Q(1$:$m,1$:$r$-$1)R( 1$:$r$-$1,r) )$

\ENDIF

\STATE $c(r)= 0 $

\STATE $k = \mathrm{max~element} \{c(r+1),\ldots,c(n)\}$

\STATE $\tau=c(k)$

\ENDWHILE

\end{algorithmic}

\caption{Pseudo code of the incomplete Cholesky factorisation process.}%
\label{algorithm:incomplete_cholesky}
\end{algorithm}


Basically, the algorithm performs an incremental solution to a $n$-hypersphere
intersection problem. The algorithm uses indices to address the data,
i.e., index $i$ is associated with pattern $\x_{i}$ from data set
$\mathcal{D}$. Two sets are introduced, being remaining set $\mathcal{R}$
and basis set $\mathcal{S}$. Coordinate matrix $\coordm$ is initially
empty. The algorithm bootstraps by computing the initial squared perpendicular
distances $d_{\perp}^{2}[i]=k(\x_{i},\x_{i})$. 

In the main loop the algorithm updates the squared perpendicular distances
according to Equation~\ref{equation:perp_distance}. By means of
these perpendicular distances, the farthest vector is chosen which
will be added to basis set $\mathcal{S}$. Coordinate matrix $\coordm$
is expanded with one column, which is updated according to 
Equation~\ref{equation:coord_update},
and set to zero at rows associated with patterns from the basis set.
Finally, the entry in the coordinate matrix associated with the latest
added basis vector is set to the last known perpendicular distance,
which is equal to the length of the last basis vector. This loop continues
until no vector is available that is farther than $\varepsilon$ from
the basis. The orthogonalisation is complete if all perpendicular
distances are zero $d_{\perp}(\sv,\nonsve_{i})=0$. 


The Incomplete Cholesky Factorisation solves a system ... such that ... . 


\section*{Refinement of}


\section*{Associated types}

\section*{Preconditions}


\section*{Complexity}

\section*{Example}


\section*{See also}




\bibliographystyle{unsrtnat}
\bibliography{/home/rutger/documents/bibliography/references}
\end{document}



