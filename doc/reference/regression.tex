
\documentclass{article}
\input{myinclude}
\usepackage[numbers]{natbib}

\begin{document}

\title{regression<T>}
\maketitle

\section*{Description}

Generally, a regression problem is to find a relation between variables. 
In machine learning, it is considered a supervised learning problem, where we 
learn from instances from a data set 
$\mathcal{D}=\{ (x_1,y_1), (x_2,y_2), \ldots, (x_n, y_n ) \}$ 
containing input-output pairs $(x_i,y_i)\in \mathcal{X}\times \mathbb{R}$. 

\begin{figure}
\includegraphics{regression_problem}
\end{figure}

\section*{Example}
\section*{Definition}

Defined in the KML header \href{\kmlsvnroot/kml/regression.hpp}{kml/regression.hpp}.

\section*{Template Parameters}
\section*{Model of}
\section*{Type requirements}
\section*{Members}
\section*{Notes}
\section*{See also}

\href{\kmlroot/reference/classification.html}{classification}, 
\href{\kmlroot/reference/ranking.html}{ranking}

\bibliographystyle{unsrtnat}
\bibliography{/home/rutger/documents/bibliography/references}
\end{document}



