
\documentclass{article}
\input{myinclude}

\begin{document}

\title{Improve}
\maketitle

\textbf{2006-12-15: Please note: documentation is undergoing a rewrite at the moment.}

\section*{News}

\begin{itemize}

\item Installed ViewCVS for better 
\href{http://www.terborg.net/cgi-bin/viewcvs.cgi/trunk/kml/}{on-line browsing}
of the KML repository.
\item For improved collaboration, a subversion repository has been 
setup at \href{http://www.terborg.net/svn/trunk/kml/}{http://www.terborg.net/svn/trunk/kml/}. A
recommended Windows client is \href{http://tortoisesvn.tigris.org/}{TortoiseSVN}.
\item Examples have been added.

\end{itemize}


The development version of the Kernel-Machine Library can be checked out anonymously from 
the trunk directory of the subversion repository located at 
\href{http://www.terborg.net/svn/trunk/kml/}{http://www.terborg.net/svn/trunk/kml/}. By using the following command
\begin{verbatim}
svn checkout http://www.terborg.net/svn/trunk/kml
\end{verbatim}
a fresh up-to-date KML directory will be obtained. 

\section*{Mailing List}
A mailing list has been setup by the library authors. 
On this list, we discuss issues like the design and implementation of the library, 
most wanted features, bugs,
and, of course, all kinds of other issues \smile.
Below, instructions are provided on how to subscribe or unsubscribe from the mailing list.

\begin{itemize}
\item \textbf{Subscribe}. If you would like to subscribe to the mailing list, please send an
e-mail to kml-devel-subscribe@terborg.net using the e-mail address you would like to subscribe
with. You are then subscribed, and will receive all e-mails sent to kml-devel@terborg.net.
\item \textbf{Unsubscribe}. If you would like to unsubscribe from the mailing list,
please send an e-mail to, you guessed it right, kml-devel-unsubscribe@terborg.net
using the e-mail address you would like to unsubscribe with. 
You will then receive a confirmation request.
\end{itemize}


\section*{Feedback}

If you do have any questions, feedback, new algorithms, or success stories, 
please do not hesitate to contact us at kml-devel@terborg.net.
We endeavour to use any kind of feedback for improvement of the library, including the
on-line documentation.


\bibliographystyle{unsrtnat}
\bibliography{/home/rutger/documents/bibliography/references}
\end{document}








