
\documentclass{article}
\input{myinclude}
\usepackage[numbers]{natbib}

\begin{document}

\title{View Matrix}
\maketitle


\section*{Description}

The view matrix class has been implemented to cover recurring operations
found in a large number of active-set type of algorithms. In algorithms
of kernel machines, the design matrix or kernel matrix is often extended
when a new support vector is added, or is shrunk when a support vector
is deleted from the active set. The standard matrix resize operation
as implemented in the uBLAS library supports a resize, but will reallocate
an entire matrix if requested. Figure~\ref{figure:view_matrix}
(right) illustrates the memory structure used by a view matrix: the
larger, static matrix will be reallocated infrequently, whereas the
view matrix may be resized in constant time $\mathcal{O}(1)$.

\begin{figure}
\fignomath{view_matrix}
\caption{An illustration of the view matrix.}
\label{figure:view_matrix}
\end{figure}


\section*{Refinement of}


\section*{Associated types}

\section*{Preconditions}


\section*{Complexity}


\section*{Example}

\highlightcpp{}
\begin{verbatim}
// our samples are input-output pairs
view_matrix< ublas::matrix<double> > my_view;

my_view.grow_row();

\end{verbatim}


\section*{See also}




\bibliographystyle{unsrtnat}
\bibliography{/home/rutger/documents/bibliography/references}
\end{document}



