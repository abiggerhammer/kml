
\documentclass{article}
\input{myinclude}
\usepackage[numbers]{natbib}

\newcommand{\half}{\tfrac{1}{2}}

\begin{document}

\title{linear<T>}
\maketitle

\section*{Description}

\texttt{Linear<T>} is a kernel function. Specifically, it is a \href{\kmlroot/mercer_kernel.html}{Mercer Kernel}.
 If k is an object of class \texttt{linear<T>}, and u and v are objects of class T, then k(u,v) returns the inner product 
%
$$k(u,v) = u^T v $$
%
which is defined to be the product if T is a scalar type. 


\section*{Example}

\highlightcpp{}
\begin{verbatim}
vector< double > u(10);
vector< double > v(10);
linear< vector< double > > kernel;
cout << kernel( u, v ) << endl;
\end{verbatim}


\section*{Definition}

Defined in the KML header \href{\kmlsvnroot/kml/linear.hpp}{kml/linear.hpp}.


\section*{Template Parameters}

\begin{tabular}{lll}
\textbf{Parameter} & \textbf{Description} & \textbf{Default} \\ 
\hline
T & The linear argument type \\ 
\end{tabular}


\section*{Model of}

\href{\kmlroot/mercer_kernel.html}{Mercer Kernel}, 
Default Constructible, Copy Constructible

\section*{Type requirements}

T must be a vector type or a numeric type; distance_squared<T> should evaluate.


\section*{Members}

\begin{tabular}{lll}
\textbf{Member} & \textbf{Where defined} & \textbf{Description} \\
\hline
\texttt{linear()} & \href{http://www.sgi.com/tech/stl/DefaultConstructible.html}{Default Constructible} & The default constructor \\
\texttt{result_type} & Input value & The type of the result: \texttt{input_value<T>} \\
\end{tabular}

\section*{Notes}

\section*{See also}

\href{\kmlroot/mercer_kernel.html}{Mercer Kernel},
\href{\kmlroot/gaussian.html}{gaussian},
\href{\kmlroot/polynomial.html}{polynomial},
\href{\kmlroot/sigmoid.html}{sigmoid}

\bibliographystyle{unsrtnat}
\bibliography{/home/rutger/documents/bibliography/references}
\end{document}



