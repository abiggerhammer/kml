
\documentclass{article}
\input{myinclude}

\begin{document}


\title{Microsoft Visual C++ Howto}
\maketitle

This page describes the necessary steps to start using 
the Kernel-Machine Library with Microsoft Visual C++.
A fully featured free version of Microsoft Visual
C++ can be downloaded from the
\href{http://msdn.microsoft.com/visualc/vctoolkit2003/}{Visual
C++ Toolkit 2003 homepage}.\\


\section{Install the Kernel-Machine Library}

This step is relatively simple: download
\href{research/kml/kml-0.1.zip}{kml-0.1.zip}
and unzip it to a directory of your choice.


\section{Install the Boost Libraries}

On Windows x86 platforms, the Boost Libraries can most easily 
be installed using self-extracting prebuilt executables.
%
\begin{itemize}
\item First, download the 
\href{http://prdownloads.sourceforge.net/boost/boost_1_32_0.exe?download}
{self-extracting executable file} from Sourceforge. 
After the download is complete, run the
executable which will then extract the files to a directory of your
choice.

\item Second, download the zip archive of the prebuilt
\href{http://prdownloads.sourceforge.net/boost/boost-jam-3.1.10-1-ntx86.zip?download}
{Boost.Jam tool}, also from Sourceforge. 
Extract the \texttt{bjam.exe} file to the Boost directory created by the first step, 
and execute it. Wait for the compilation process to finish.
\end{itemize}
%
The Boost libraries are now installed. If anything went wrong, then more detailed information about 
Boost can be found at  \href{http://www.boost.org/}{Boost's homepage}.



\section{Install Boost Numeric Bindings}

Download \href{research/kml/boost\_bindings.zip}{boost\_bindings.zip}
and unzip it into the Boost directory. Make sure the subdirectories are aligned, 
i.e. the directory stucture 
\texttt{boost\textbackslash{}numeric\textbackslash{}bindings} should exist. 


\section{Install ATLAS Binaries}

Dynamic loadable libraries of ATLAS can be obtained in two ways: by using 
convenient prebuilt libraries provided here (if available), or by creating DLLs
of ATLAS by hand. 

\subsection{Prebuilt ATLAS libraries}

The following files are available here: 
\href{research/kml/WinNT\_ATHLONSSE1.zip}{WinNT\_ATHLONSSE1.zip}
and
\href{research/kml/WinNT\_PIIISSE1.zip}{WinNT\_PIIISSE1.zip}
. If you use these, you can skip the next subsection and go ahead to 
section \ref{section:using_kml}.

If your architecture is not listed here, and if you have successfully 
built an ATLAS dll and lib (as described below), please send us a copy of the dll and
lib file. It will then be provided here as well.




\subsection{Build your own ATLAS binaries}




\subsubsection{Obtain the Software}

\begin{itemize}
\item First, download and unzip the ATLAS 
\href{http://prdownloads.sourceforge.net/math-atlas/atlas3.6.0.tar.gz?download}
{platform-independent sources}
from Sourceforge. 
\item Apply the patches ``\href{http://math-atlas.sourceforge.net/errata.html#winsse}
{Assembler renaming problem for Windows machine}'' and the ``
\href{http://math-atlas.sourceforge.net/errata.html#longcomp}
{String overrun in config for long compiler paths}'', 
described at the 
\href{http://math-atlas.sourceforge.net/errata.html}
{ATLAS errata page}. There patches need to be applied, otherwise the
ATLAS build process will most probably fail.

\item Download and install Cygwin from the
\href{http://www.cygwin.com/}{Cygwin homepage}. Add the
following additional packages to the default install:
Devel->gcc-g++, Devel->gcc-g77 and Devel->make.

\end{itemize}

\subsubsection{Compile ATLAS}

Open a Cywin console window, and change to the ATLAS root
directory, and start the configuration of ATLAS.

\begin{verbatim}
cd C:/ATLAS
make config
\end{verbatim}
%
In this step ATLAS wants to figure out more about your CPU.
If you are not sure about the exact specifics of your processor, then
you could try a hardware detection program, such as 
\href{http://www.cpuid.org/cpuz.php}{CPU-Z}.

You will be asked a series of questions, shown below. If you press enter 
without answering, it will use the [default] value. 
%
\begin{verbatim}
Enter number at top left of screen [0]: 24
Have you scoped the errata file? [y]:
Are you ready to continue? [y]:
Enter your machine type:
   1. Other/UNKNOWN
   2. AMD Athlon
   3. 32 bit AMD Hammer
   4. 64 bit AMD Hammer
   5. Pentium PRO
   6. Pentium II
   7. Pentium III
   8. Pentium 4
Enter machine number [1]: 2
enable Posix  threads support? [n]: 
use express setup? [y]:
Enter Architecture name (ARCH) [WinNT_ATHLONSSE1]:
Enter Maximum cache size (KB) [4096]:
Enter File creation delay in seconds [0]:
Use supplied default values for install? [y]:
\end{verbatim}
%
After these questions have been answered, ATLAS will create makefiles on the basis of your
configuration. It will give you a confirmation of success. You can now issue the next 
command, which is something like 
%
\begin{verbatim}
make install arch=WinNT_ATHLONSSE1
\end{verbatim}
%
This will start the ATLAS compilation process. After compiling, 
you will have a subdirectory called \texttt{lib/WinNT\_ATHLONSSE1} 
in your ATLAS directory, which should contain the following files: 
\texttt{libatlas.a}, \texttt{libcblas.a}, \texttt{libf77blas.a}, and 
\texttt{liblapack.a}.


\subsubsection{Create the dll and lib}

Download 
\href{research/kml/kml\_win\_dll.sh}{kml\_win\_dll.sh}
to the directory that contains the \texttt{libatlas.a}, etc., files.  
(Re-)Open a Cygwin console, and change to that directory. Run the command
\begin{verbatim}
./kml_win_dll.sh
\end{verbatim}
%
You will get a message ``Generating
the KML ATLAS DLL''. After this command is complete, you should be able to see new 
files called \texttt{kml\_atlas.dll} and \texttt{kml\_atlas.lib} in that directory.


\section{Link to the DLL and lib in Visual C++}
\label{section:using_kml}

In order to link against the ATLAS routines, you need to do two things in 
Visual Studio.
First, put the \texttt{kml\_atlas.dll} file in one of your path
directories: the directory the executable of your
own program resides will do. Second, instruct the linker to link your 
project against \texttt{kml\_atlas.lib}.



\end{document}

