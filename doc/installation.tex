
\documentclass{article}
\input{myinclude}
\usepackage[numbers]{natbib}


\begin{document}

\title{Installation Guide}
\maketitle

\textbf{2006-12-20: Please note: documentation is undergoing a rewrite at the moment.}

\tableofcontents


\section{Dependencies}

The Kernel-Machine Library builds on top of the libraries below.

\begin{itemize}
\item \textbf{Boost.} Functionality of several \href{http://www.boost.org/}{Boost libraries} is 
in use throughout the Kernel-Machine Library. 
Boost provides
an excellent set of C++ libraries which reduces the amount of code and boosts 
the quality of the overall implementation.
The Kernel-Machine Library was originally built and tested against version 1.32.0, and also works 
with the current version of 1.33.1. 

\item \textbf{ATLAS.} This is a very efficient basic linear algebra system 
(\href{http://www.netlib.org/blas/faq.html}{BLAS}) \cite{whaley01automated}, located at
\href{http://math-atlas.sourceforge.net}{Sourceforge}. 
ATLAS can empirically finetune 
its code to make optimal use of your CPU 
including its numerical extensions such as SSE, 3DNow! and AltiVec. 
It also contains several LAPACK routines. 

\item \textbf{Boost Numeric Bindings.} A part of Boost's sandbox CVS repository is 
included in the KML distribution, 
namely the numeric bindings. 
This is a software package that combines the flexibility of the C++ data types (such as that of 
\texttt{std::vector<>} or \texttt{ublas::matrix<>}) with the computational 
efficiency of a BLAS and LAPACK. The numeric bindings are .

\end{itemize}


\section{Debian GNU/Linux}

This section describes how to start developing with
the Kernel-Machine library on a Debian GNU/Linux-based operating system.

Using root privileges, please run \texttt{apt-get} to install the following packages.
\begin{verbatim}
$ apt-get install atlas3-base-dev g++ scons subversion libboost-dev \ 
libboost-serialization-dev libboost-program-options*
\end{verbatim}
%
Using your local user account, execute the following commands.
\begin{verbatim}
$ cd ~/my/development/dir/
$ svn checkout http://www.terborg.net/svn/trunk/kml
$ cd kml
$ scons
\end{verbatim}
%
By executing the \texttt{scons} command, some examples will be built the in the directory \texttt{kml/examples/}.
If this is your first install, perhaps you are interested in continuing 
by reading the \href{\kmlroot/tutorial.html}{tutorial}, or browsing through 
the \href{\kmlroot/reference/}{Reference}.

If for some reason these instructions did not work out as advertised here,
please drop us a message at kml-devel@terborg.net, and/or take a look
at the \href{\kmlroot/improve.html}{improvement guide}.

% Suppose you would like to check for updates of the software,
% basically performing a   
% %
% \begin{verbatim}
% /kml$ svn update
% \end{verbatim}
% %
% will update your local repository.

% \section{Download}
% \label{section:download}
% 
% The latest release of the Kernel-Machine Library is 0.1. However, this is not the recommended version; you are encouraged to
% download the latest developer version through  
% \href{http://www.terborg.net/cgi-bin/viewcvs.cgi/trunk/kml/}{viewCVS} until 0.2 is released. 
% Version 0.1 can still be downloaded, either as a tarball 
% from \href{research/kml/kml-0.1.tar.gz}{kml-0.1.tar.gz} or as a zipped archive
% from \href{research/kml/kml-0.1.zip}{kml-0.1.zip}.
% 
% I have created a tarball of 
% \href{research/kml/boost\_bindings.tar.gz}{boost\_bindings.tar.gz}
%  and a zipfile 
% \href{research/kml/boost\_bindings.zip}{boost\_bindings.zip}%
% , which are on basis of Boost's sandbox CVS of April 3rd, 2005.
% 
% 
% The development version of the Kernel-Machine Library can be checked out anonymously from 
% the trunk directory of the subversion repository located at 
% \href{http://www.terborg.net/svn/trunk/kml/}{http://www.terborg.net/svn/trunk/kml/}. By using the following command
% \begin{verbatim}
% svn checkout http://www.terborg.net/svn/trunk/kml
% \end{verbatim}
% a fresh up-to-date KML directory will be obtained. 


\section{Microsoft Windows}

This section describes the necessary steps to start using 
the Kernel-Machine Library on a Microsoft Windows System, 
using free software only. To be able to perform these
steps, you will need Administrator privileges.

\subsection{Install Visual C++}

To develop applications comfortably under Microsoft Windows, you will need
to have installed both a \href{http://msdn.microsoft.com/vstudio/express/visualc/}{C++ compiler},
and the 
\href{http://msdn.microsoft.com/vstudio/express/visualc/usingpsdk/}{Windows Platform SDK} (the latter is
already included in the non-free version of Visual C++).

Some additional installation instructions and hints are also available the notes
on the page about Boost's
\href{http://www.boost.org/tools/build/v1/vc-8_0-tools.html}{vc-8_0 toolset}.

To check whether the compiler is installed, please run the Visual C++ Command Window (should be 
available in your menu), and try to execute \texttt{cl}. Check whether 
a file \texttt{vcvars32.bat} exists in the \texttt{bin} subdirectory of the Visual
C++ compiler, as it is needed to build the Boost Libraries in the next step.

\subsection{Install the Boost Libraries}

On Windows x86 platforms, the Boost Libraries can most easily 
be installed using self-extracting prebuilt executables.
The authors of the Boost libraries provide installation instructions at 
their \href{http://www.boost.org/more/getting_started.html}{getting started} page.

For the impatient, we have provided the following steps.
%
\begin{enumerate}
\item Download the self-extracting executable file
\href{http://prdownloads.sourceforge.net/boost/boost_1_33_1.exe?download}
{boost_1_33_1.exe} from Sourceforge. 
After the download is complete, run the
executable, and extract the files to \texttt{C:\textbackslash Boost}.

\item Download the zip archive of 
\href{http://prdownloads.sourceforge.net/boost/boost-jam-3.1.13-1-ntx86.zip?download}
{boost-jam-3.1.13-1-ntx86.zip}, also from Sourceforge. 
Extract the \texttt{bjam.exe} file to the directory \texttt{C:\textbackslash Boost} created in step 1.

\item Open the Visual C++ Command Window, change directory to \texttt{C:\textbackslash Boost}, 
type
\begin{verbatim}
bjam "-sTOOLS=vc-8_0"
\end{verbatim}
and wait for the compilation process to finish.

\end{enumerate}
%
The Boost libraries are now installed. If anything went wrong, then more detailed information about 
Boost can be found at  \href{http://www.boost.org/}{Boost's homepage}.

% \subsection{Install Boost Numeric Bindings}
% 
% Download \href{research/kml/boost\_bindings.zip}{boost\_bindings.zip}
% and unzip it into the Boost directory. Make sure the subdirectories are aligned, 
% i.e. the directory stucture 
% \texttt{boost\textbackslash{}numeric\textbackslash{}bindings} should exist. 


\subsection{Install ATLAS}

Obtaining working ATLAS binaries under Microsoft Windows could be one of the
most annoying steps. 

If you are lucky, you could use 
convenient prebuilt libraries provided here (if available). Otherwise 
you will have to create ATLAS DLLs
manually. 

\subsubsection{Prebuilt ATLAS libraries}

The following files are available here:
\begin{itemize}
\item \href{research/kml/WinNT\_ATHLONSSE1.zip}{WinNT\_ATHLONSSE1.zip}
\item \href{research/kml/WinNT\_PIIISSE1.zip}{WinNT\_PIIISSE1.zip}
\end{itemize}
If you use these, you can skip the next subsection and go ahead to 
subsection \ref{section:install_scons}.

If your architecture is not listed here, and if you have successfully 
built an ATLAS dll and lib (as described below), please send us a copy of the dll and
lib file at kml-devel@terborg.net. Your hard work will safe 
other people's time, which we think is a good thing.

\subsubsection{Build your own ATLAS binaries}

\begin{enumerate}
\item Download the platform-independent source package
\href{http://prdownloads.sourceforge.net/math-atlas/atlas3.6.0.tar.gz?download}{atlas3.6.0.tar.gz}
from Sourceforge, and unzip it to \texttt{C:\textbackslash Atlas}.

\item Apply the patches ``\href{http://math-atlas.sourceforge.net/errata.html#winsse}
{Assembler renaming problem for Windows machine}'' and the 
``\href{http://math-atlas.sourceforge.net/errata.html#longcomp}{String overrun in config for long compiler paths}'',
described at the 
\href{http://math-atlas.sourceforge.net/errata.html}
{ATLAS errata page}. Both patches need to be applied, otherwise the
ATLAS build process will most probably fail.

\item Install Cygwin by running the 
\href{http://www.cygwin.com/setup.exe}{setup.exe}. Make sure to have added the
following packages to the default install:
\begin{itemize}
% 2006-12-18. The right-arrow issue.
% first tried to use \usepackage{textcomp} in combination with \textrightarrow{}; which produces
% the correct result: an utf-8 encoded rightarrow. Unfortunately, this character is not (yet) supported 
% on enough browsers. Therefore, a math-symbol is inserted. :-(
\item Devel $\rightarrow$ gcc-g++
\item Devel $\rightarrow$ gcc-g77
\item Devel $\rightarrow$ make
\end{itemize}

\item Gather information about the cache size of your CPU.
If you are not sure about the exact specifics of your processor, then
you could try a hardware detection program, such as 
\href{http://www.cpuid.org/cpuz.php}{CPU-Z}.

\item Open a Cywin console window, and change to the ATLAS root
directory, and start the configuration of ATLAS.
%
\begin{verbatim}
cd C:\Atlas
make config
\end{verbatim}
%
You will be asked a series of questions, shown below. If you press enter 
without answering, it will use the [default] value. 
%
\begin{verbatim}
Enter number at top left of screen [0]: 24
Have you scoped the errata file? [y]:
Are you ready to continue? [y]:
Enter your machine type:
   1. Other/UNKNOWN
   2. AMD Athlon
   3. 32 bit AMD Hammer
   4. 64 bit AMD Hammer
   5. Pentium PRO
   6. Pentium II
   7. Pentium III
   8. Pentium 4
Enter machine number [1]: 2
enable Posix  threads support? [n]: 
use express setup? [y]:
Enter Architecture name (ARCH) [WinNT_ATHLONSSE1]:
Enter Maximum cache size (KB) [4096]:
Enter File creation delay in seconds [0]:
Use supplied default values for install? [y]:
\end{verbatim}
%
After these questions have been answered, ATLAS will create makefiles on the basis of your
configuration. It will give you a confirmation of success. 

\item Start the build process, which is initiated by something like 
%
\begin{verbatim}
make install arch=WinNT_ATHLONSSE1
\end{verbatim}
%
depending on the name of your processor. After compiling, 
you will have a subdirectory called \texttt{lib/WinNT\_ATHLONSSE1} 
in your ATLAS directory, which should contain the following files:
\begin{itemize}
\item \texttt{libatlas.a}
\item \texttt{libcblas.a}
\item \texttt{libf77blas.a}
\item \texttt{liblapack.a}
\end{itemize}
These files are \emph{the} ATLAS binaries, 
and they may be used to create a DLL containing these files.

\item This step is optional: it is only needed if you are 
\textbf{not} installing the Kernel-Machine Library, but rather
would like to obtain a stand-alone ATLAS installation.
Download 
\href{research/kml/kml\_win\_dll.sh}{kml\_win\_dll.sh}
to the directory that contains the \texttt{libatlas.a}, etc., files.  
(Re-)Open a Cygwin console, and change to that directory. Run the command
%
\begin{verbatim}
./kml_win_dll.sh
\end{verbatim}
%
You will get a message ``Generating
the KML ATLAS DLL''. After this command is complete, you should be able to see new 
files called
\begin{itemize}
\item \texttt{kml\_atlas.dll}
\item \texttt{kml\_atlas.lib}
\end{itemize}
in that directory.
%
If you would like to link against the ATLAS routines from within Visual Studio, 
you will have to do two things in Visual Studio:
\begin{itemize}
\item ensure the \texttt{kml\_atlas.dll} file is located in your PATH
\item instruct the linker to link your project against \texttt{kml\_atlas.lib}.
\end{itemize}


\end{enumerate}

\subsection{Install SCons}
\label{section:install_scons}

Prior to installing SCons, install Python
by downloading and running the installer \href{http://www.python.org/ftp/python/2.5/python-2.5.msi}{python-2.5.msi}.
Make sure that Python is in your PATH, e.g., by adding by \texttt{C:\textbackslash Python25} 
to your PATH environment variable, as
described \href{http://www.python.org/doc/faq/windows.html#how-do-i-run-a-python-program-under-windows}{here}.

Next, install Scons by running 
\href{http://prdownloads.sourceforge.net/scons/scons-0.96.1.win32.exe?download}{scons-0.96.1.win32.exe}. 
The installation process will find
Python automatically.

\subsection{Install the Kernel-Machine Library}
Download the latest zipped archive of the Kernel-Machine Library, 
\href{http://www.terborg.net/cgi-bin/viewcvs.cgi/trunk/kml.tar.gz?view=tar}{kml.tar.gz}, 
and unzip it to a directory of your choice, e.g., \texttt{C:\textbackslash kml}.

Open a Visual C++ Command Window, and execute the following steps.
%
\begin{verbatim}
cd C:\kml
scons
\end{verbatim}
%
By executing the \texttt{scons} command, the ATLAS DLL will be built, 
and some examples will be built in the directory \texttt{kml\textbackslash examples}.
If this is your first install, perhaps you are interested in continuing 
by reading the \href{\kmlroot/tutorial.html}{tutorial}, or browsing through 
the \href{\kmlroot/reference/}{Reference}.

If for some reason these instructions did not work out as advertised here,
please drop us a message at kml-devel@terborg.net, and/or take a look
at the \href{\kmlroot/improve.html}{improvement guide}.

\bibliographystyle{unsrtnat}
\bibliography{/home/rutger/documents/bibliography/references}

\end{document}

