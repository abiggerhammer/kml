
\documentclass{article}
\input{myinclude}

\usepackage[numbers]{natbib}

\begin{document}

\title{Kernel-Machine Library}
\maketitle



\section*{Introduction}

The Kernel-Machine Library is a freely available (released under the 
\href{http://www.gnu.org/copyleft/lesser.html}{LGPL}) C++ library to
promote the use and progress of kernel machines. 
It is both for academic use and for developing real world applications. 
The Kernel-Machine Library draws heavily from features of modern C++ such 
as template meta-programming to achieve high 
performance while at the same time offering a comfortable interface. 
It enables compile-time selection of specialised algorithms on the basis of data types: for example, 
the specific case of a SVM in combination with a linear kernel can be computed by a 
specialised efficient algorithm. 

The Kernel-Machine Library has implementations for the following kernel machines and their cited
algorithms:
%
\begin{itemize}
\item Support Vector Machine \citep{platt99fast,ma03accurate,engel02sparse}
\item Relevance Vector Machine \citep{tipping03fast}
\item Kernel Recursive Least Squares \citep{engel03kernel}
\item Adaptive Sparseness using Jeffreys Prior \citep{figueiredo03adaptive}
\item Smooth Relevance Vector Machine \citep{terborg05smooth}
\end{itemize}
%
Until now, the focus has been on regression. The handling of classification and ranking problems 
is being added.


\section*{News}

\begin{itemize}

\item Installed ViewCVS for better 
\href{http://www.terborg.net/cgi-bin/viewcvs.cgi/trunk/kml/}{online browsing}
of the KML repository.
\item A developer mailing list has been setup. If you would like to subscribe, sending an
e-mail to kml-devel-subscribe@terborg.net will suffice.
\item Switched to the excellent \href{http://www.scons.org/}{SCons} build system for 
improved cross-platform development.
\item For improved collaboration, a subversion repository has been 
setup at \href{http://www.terborg.net/svn/trunk/kml/}{http://www.terborg.net/svn/trunk/kml/}. A
recommended Windows client is \href{http://tortoisesvn.tigris.org/}{TortoiseSVN}.
\item Examples have been added.

\end{itemize}

\section*{Dependencies}

Besides a few small bugs and quirks, the library has been successfully tested 
on GNU/Linux, Windows 2000 and Windows XP. You will need a reasonably
standards-compliant compiler such as GCC or MS VC++ (7.1 or higher).  
The Kernel-Machine Library builds on top of the libraries below.

\begin{itemize}
\item \textbf{Boost.} Functionality of several \href{http://www.boost.org}{Boost libraries} is 
in use throughout the Kernel-Machine Library. Boost provides
an excellent set of C++ libraries which reduces the amount of code and boosts the quality of the overall implementation.
The Kernel-Machine Library is built and tested against version 1.32.0.

\item \textbf{Boost Numeric Bindings.} A part of Boost's sandbox CVS repository is required, namely the numeric bindings. 
This is a software package that combines the flexibility of the C++ data types (such as that of 
\texttt{std::vector<>} or \texttt{ublas::matrix<>}) with the computational 
efficiency of a BLAS and LAPACK.  

\item \textbf{ATLAS.} This is a very efficient basic linear algebra system (BLAS) \citep{whaley01automated}, located at
\href{http://math-atlas.sourceforge.net}{Sourceforge}. ATLAS can empirically finetune its code to make optimal use of your CPU 
including its extensions such as SSE, 3DNow! and AltiVec. It also contains several LAPACK routines. 

\end{itemize}



\section*{Obtaining the Library}
\label{section:downloads}

The latest release of the Kernel-Machine Library is 0.1. It can be downloaded as a tarball 
from \href{research/kml/kml-0.1.tar.gz}{kml-0.1.tar.gz} or as a zipped archive
from \href{research/kml/kml-0.1.zip}{kml-0.1.zip}.

I have created a tarball of 
\href{research/kml/boost\_bindings.tar.gz}{boost\_bindings.tar.gz}
 and a zipfile 
\href{research/kml/boost\_bindings.zip}{boost\_bindings.zip}%
, which are on basis of Boost's sandbox CVS of April 3rd, 2005.


The development version of the Kernel-Machine Library can be checked out anonymously from 
the trunk directory of the subversion repository located at 
\href{http://www.terborg.net/svn/trunk/kml/}{http://www.terborg.net/svn/trunk/kml/}. By using the following command
\begin{verbatim}
svn checkout http://www.terborg.net/svn/trunk/kml
\end{verbatim}
a fresh up-to-date KML directory will be obtained. 

\section*{Documentation}

Documentation describing the design and APIs of the Kernel Machine Library 
in doxygen format can be found \href{research/kml/doxygen/}{here}.

Below are installation instructions for installing the Kernel-Machine Library on 
Windows.

\begin{itemize}

\item A convenient step-by-step 
\href{research/kml/installation.html}{Microsoft Visual C++ Howto} 
to start using the Kernel-Machine Library with Microsoft Visual C++ has been 
contributed by Thijs van den Berg.

\end{itemize}


\section*{Feedback}

If you do have any questions, feedback, new algorithms, or success stories, 
please don't hestitate to contact me at rutger@terborg.net.


\bibliographystyle{unsrtnat}
\bibliography{/home/rutger/documents/bibliography/references}
\end{document}








