
\documentclass{article}
%\input{myinclude}

\begin{document}

\title{Tutorial}
\maketitle
%\tableofcontents

\textbf{2006-12-29: Please note: documentation is undergoing a rewrite at the moment.}


\section*{Sinc regression}

A common benchmark used in the kernel-machine literature is that of a toy sinc function. 
Here we show how to reproduce these examples with a kernel machine of your choice. 
First, let us define our data structure.
%
\highlightcpp{}
\begin{verbatim}
#include <boost/vector_property_map.hpp>
typedef std::pair< double, double > example_type;
typedef boost::vector_property_map< example_type > container_type;
\end{verbatim}
%
The \texttt{example_type} is an input-output pair $(x_i,y_i)$ with both inputs and outputs being
of type \texttt{double}.
The \texttt{container_type} provides a type definition for the data container to
be used by the kernel machine of choice 
(this could also very well be, e.g., a custom class that 
accesses a database on-the-fly to retrieve examples of example_type).

We let the data for one experiment consist of 50 examples $(x_1,y_1),\ldots,(x_{50},y_{50})$,
with inputs $x_{i}$ uniformly distributed over [-10,10],
and outputs corrupted with Gaussian noise, $y_{i}\sim N(\mathrm{sinc}(x_{i}),0.1)$.
%
\highlightcpp{}
\begin{verbatim}
#include <boost/random/mersenne_twister.hpp>
#include <boost/random/normal_distribution.hpp>
#include <boost/random/variate_generator.hpp>
#include <boost/math/special_functions/sinc.hpp>

// create the random number generator N(0,0.1)
boost::mt19937 randomness;
typedef boost::normal_distribution<double> dist_type;
dist_type norm_dist( 0.0, 0.1 );
boost::variate_generator< boost::mt19937, dist_type > noise( randomness, 
                                                             norm_dist );
// instantiate the data container, and fill it with (x,y)-pairs
container_type my_data;
for( int i=0; i<50; ++i ) {
  double x = static_cast<double>(i) / 49.0 * 20.0 - 10.0;
  my_data[ i ] = std::make_pair( x, boost::sinc(x) + noise() );
}
\end{verbatim}
%
To generate the outputs, we have pulled in parts from the 
\href{http://www.boost.org/libs/random/index.html}{Boost Random Number Library}, and the
sinc function from the
\href{http://www.boost.org/libs/math/special_functions/index.html}{Boost Math Special Functions Library}. 
%
%
Figure~\ref{figure:sinc_regression} shows an example scatterplot of the data that we 
just built the generator for. 

\begin{figure}
\includegraphics{sinc_data}
\caption{Input data generated for our first tutorial example.}
\label{figure:sinc_regression}
\end{figure}


% 
% \section*{Writing your own kernel}
% 
% In this part of the tutorial we will create a novel type of kernel: a string kernel. To do so, 
% simply create a fresh source-file, called, e.g., string_kernel.hpp. 
% 
% \highlightcpp{}
% \begin{verbatim}
% class string_kernel: public kml::kernel< std::string, double, 
%                                          string_kernel, kml::mercer_tag > {
% };
% \end{verbatim}
% 
% The first template parameter indicates that the inputs we are accepting are of type std::string. The second
% template parameter says that our kernel function will return doubles, the third uses a C++ pattern called
% the \emph{Curiously Recurring Template Pattern}, which enables some tricks, and the fourth parameter indicates
% that we are constructing a string kernel.
% 
% \highlightcpp{}
% \begin{verbatim}
% // evaluate the "closeness" of strings u and v
% double dot( std::string const &u, std::string const &v ) const {
% ...
% }
% \end{verbatim}
% 
% If your kernel provides parameters, it should encode that in some way such that the kernel-machine 
% libary may use those. ... 
% 
% Now, let's see what we have gained with this new kernel.
% 
% \highlightcpp{}
% \begin{verbatim}
% string_kernel my_kernel( sigma=2.0 );
% kml::svm< ..., string_kernel, > my_machine;
% my_machine.learn( keys.begin(), keys.end() );
% \end{verbatim}
% 
% This will deliver a SVM with your new kernel compiled in. 
% 
% 
% \section*{Data I/O}
% 
% Kernel machines are used a lot for data mining, and given that data mining involves data, the IO functionaly
% of the kernel-machine library is very convenient. It allows for the storage of all types that have serialization 
% support. 
% 
% \highlightcpp{}
% \begin{verbatim}
% std::vector< std::size_t > my_keys;
% boost::property_map< ... > my_data;
% kml::file my_file( "somedata.txt" );
% // read file which is in text form
% my_file.read( my_data );
% // try to convert to binary first, then read the file
% my_file.read( my_data, kml::io::convert_to_binary );
% my_file.close();
% \end{verbatim}
% 
% All kernel machines available in the KML have serialization support, which means that you can load and save them 
% accordingly.
% 
% \highlightcpp{}
% \begin{verbatim}
% std::ofstream my_file( "outputfile.txt" | std::ios::binary );
% boost::binary_ofstream ofs( my_model );
% ofs << my_machine;
% my_file.close();
% \end{verbatim}
% 
% \section*{Writing your own kernel machine}
% 
% 
% \section*{Reinforcement Learning}
% 
% For something different, let us implement the reinforcement learning as done by REF...  
% We will implement a SARSA algorithm using kernel machines to make things interesting.

\section*{See Also}
\href{\kmlroot/reference/}{Reference}


\bibliographystyle{unsrtnat}
\bibliography{/home/rutger/documents/bibliography/references}
\end{document}

